\documentclass[../main.tex]{subfiles}
\graphicspath{{\subfix{../images/}}}

\begin{document}

\subsection{Input Devices}

Input devices detect actions from the user. For example, a touchscreen detects the action of the user touching the screen, and a button detects the action of the user pressing the switch down. Examples of input devices include:

\begin{itemize}
    \item \textbf{Barcode Scanners}, which scans barcodes
    \item \textbf{QR Code Scanners}, which scans QR codes
    \item \textbf{Keyboards}, for text input
    \item \textbf{Mice (Optical Mice)}, which gives user precise position based inputs into a computer.
    \item \textbf{Touchpads/Trackpads}, which does the same thing as a mouse, but it revolves around the user moving their finger on a sensitive surface.
    \item \textbf{Digital Cameras}, which capture the outside world as binary data
    \item \textbf{Microphones}, which record sound data and convert it into digital sound waves.
    \item \textbf{Scanners} captures 2D images or 3D models of a physical object in a precise way.
        \begin{itemize}
            \item 2D scanners scan 2D documents and such to create a typically black-and-white image that is sharp. Similar to a camera, but works best for documents.
            \item 3D scanners employ complex technology to create 3D models of objects you place in it.
        \end{itemize}
    \item \textbf{Touchscreens}, which allows the user to touch the screen itself to allow less-precise but human gesture controls for a computer/phone/tablet.
        \begin{itemize}
            \item Resistive Touchscreens have a flexible layer on top of the display. When the user presses the screen, the bottom of the display can detect where the user pressed on the flexible layer.
            \item Capacitive Touchscreens detect touches by seeing if there is a conductive material on the surface of it, which is electrically charged. Since your finger is very minimally conductive, it can detect it as a touch.
            \item Infrared Touchscreens detect touches by shining a row and column of closely-packed infrared lasers. When your finger blocks 2 beams, the x and y coordinates are sent to the computer.
        \end{itemize}
\end{itemize}


Note that details about the devices themselves arent provided. Please research on your own or refer to the textbook.

\subsection{Output Devices}

Output devices allows a computer system to create a tangible reaction or an event in the real world, like displaying data or starting a water pump.

\begin{itemize}
    \item \textbf{Displays}, or screens are found in laptops, TVs, tablets, phones, etc. Their role is obvious. 
    \item \textbf{Projectors} are like displays, but they generate light beams that shine the computer's output into a surface, like a wall, whiteboard or a projector screen.
    \item \textbf{Actuators}, including things like solenoids and pumps creates motion that is controlled by a computer. As an example, a solenoid is like a piston; it pushes things.
    \item \textbf{Printers} convert image data from a computer\footnote{Fun fact: text must be rendered as pixels before it can be printed, so it is image data!} and puts it onto paper. There are 3 types of printers.
        \begin{itemize}
            \item Laser Printers shine a laser onto toner, which is a very fine powder that acts as a dye onto a piece of paper, basically burning it into the paper. Almost all workplaces and schools uses them.
            \item Inkjet Printers shoot very small jets of ink onto paper. They produce better colors than laser, which is why they are used more in art institutions.
            \item Dot-matrix Printers are very old and typically only used in airports. They are the loudest, and make a high pitched screeching noise. You may have seen them near boarding gates; they punch very shallow holes and fill them with ink in a matrix of dots. The resolution and quality is low.
        \end{itemize}
    \item \textbf{Speakers} broadcast computerized binary sound data.
    \item \textbf{3D printers} are like 2D printers, but they use a filament that is heated up to produce accurate reproductions of 3D models in a computer/
\end{itemize}

\subsection{Sensors}

Sensors detect changes in the surrounding environment, i.e. they \emph{sense} the things happening around it. As data in the real world is always analog, it must always go through an ADC (Analog to Digital Converter, \emph{NOT a DAC!})

\begin{itemize}
    \item \textbf{Temperature Sensors} sense temperature.
    \item \textbf{Humidity Sensors} sense the concentration of water in the air.
    \item \textbf{Acoustic Sensors} detect specific sounds, like the sound of glass breaking in a criminal-infested museum.
    \item \textbf{Pressure Sensors} detect how hard an object has been pressed. Pressure sensors that detect the pressure of a container do also exist, but they are usually not digital.
    \item \textbf{Gas Sensors} detects the concentration of a specific gas in the air.
    \item \textbf{An accelerometer} detects the acceleration of an object, like sudden shakes or movements of a phone.
    \item \textbf{A Gyroscope} detects the movement of an object in space, including forward and backward movements, up and down movement and rotation. In essence, a more precise accelerometer.
    \item \textbf{An active infrared sensor} detects if beams of infrared light are blocked or not. A signal will be sent out if it is.
    \item \textbf{A passive infrared sensor} detects IR heat radiation.
\end{itemize}

\end{document}
