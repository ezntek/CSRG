\documentclass[../main.tex]{subfiles}
\usepackage[hmargin=2.5cm,vmargin=3cm,bindingoffset=0.5cm]{geometry}
\graphicspath{{\subfix{../images/}}}

\begin{document}

\enumerate

\item This revision text is authored by Eason Qin Luojia, with contributors listed on the cover page; including and not limited to Siddharth Harish ando Karthik Sankar.

\item Some excercises for the Chapter Ten content on Logic Gates may be pulled from the textbook\footnote{Cambridge IGCSE™ and O Level Computer Science, Second Edition, ISBN 9781398318281}, but some are also generated by the authors.

\item \textbf{Formal Citations and a bibliography are not provided}, as this is not an academic research document, but a reference booklet of notes from the IGCSE Computer Science 0478 course offered at my school, along with content from the textbook (as mentioned previously). Since the work is mostly produced from either directly pulling examples from the textbook (which will be annotated) or already synthesized information, no references for those points will be provided. If there is information that \textit{must} be cited, including and not limited to extremely detailed data points, the source will be provided as a foot note. \textbf{In no case will MLA, APA, Harvard or any other form of formal academic referencing be used.}\footnote{Legally, all licenses will be followed; i.e. if the document has a license that requires attribution, the attrbution will be provided, etc.}

\item If there is an underlined portion of text, like so:

      \textit{Chatbots have mostly been replaced by LLMs, simply go to the AI section below}

      is seen, and you have the \textbf{printed copy}, simply go to the section it says; do so via the \ref{table-of-contents}.


\endenumerate

\end{document}
