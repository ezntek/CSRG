\documentclass[../main.tex]{subfiles}
\graphicspath{{\subfix{../images/}}}

\begin{document}

\subsection{File Size Units}

The smallest unit of size in Computer Science is a \textbf{bit} (binary digit), and it is either a 1 or a 0. A byte is 8 bits, and a nibble is 4 (one hexadecimal digit). The table below introduces to you the rest of the units you need to know.

\begin{table}[H]
    \centering
    \begin{tabular}{|c|c|}
        \hline
        \textbf{Number of Bytes} & \textbf{Unit}
        \\ \hline
        1000 & Kilobyte (KB)
        \\ \hline
        1,000,000 & Megabyte (MB)
        \\ \hline
        1,000,000,000 & Gigabyte (GB)
        \\ \hline
        1,000,000,000,000 & Terabyte (TB)
        \\ \hline
        1,000,000,000,000,000 & Petabyte (PB)
        \\ \hline
    \end{tabular}
    \caption{SI Base-10 based file size units}
    \label{tab:base10_units}
\end{table}

Table \ref{tab:base10_units} shows the different units if going in increments of 1000. This is based on base-10 and is easier for humans to understand. However, since computers operate in base-2, it makes much more sense for them to go in increments of 1024, like so.

\begin{table}[H]
    \centering
    \begin{tabular}{|c|c|}
        \hline
        \textbf{Number of Bytes} & \textbf{Unit} 
        \\ \hline
        1024 ($2^{10}$) & Kibibyte (KiB)
        \\ \hline
        1,048,576 ($2^{20}$) & Mebibyte (MiB)
        \\ \hline
        1,073,741,824 ($2^{30}$) & Gibibyte (GiB)
        \\ \hline
        1,099,511,627,776 ($2^{40}$) & Tebibyte (TiB)
        \\ \hline
        1,125,899,906,842,624 ($2^{50}$) & Pebibyte (PiB)
        \\ \hline
    \end{tabular}
    \caption{SI Base-10 based file size units}
    \label{tab:computer_units}
\end{table}

This means that data can be represented in ($2^{10}$), ($2^{20}$), etc. which is more accurate, as devices measure memory in powers of 2, as they store binary bytes and not decimal 10s. This means that 1 Terabyte is only \textasciitilde931 Gibibytes.

\subsection{Compression}

There are 2 main types of compression.

\paragraph{Lossy}

This means that unnecessary data is eliminated from the file to reduce its size. Examples include:

\begin{itemize}
    \item \textbf{JPEG}, which throws away some data (image quality/sharpness) to reduce file size. This may change the bit depth, etc.
    \item \textbf{MP3/MP4}, which are audio and video formats respectively that may throw away sampling rate, resolution and image quality for a video to save on file size. 
\end{itemize}

\paragraph{Lossless (Run-length Encoding)}

This means that no data is discarded to reduce the file size. RLE is a method in the syllabus you must know that collapses repeating sequences of the same data into one chunk of data. Consider the string {\mono aaaaabbbccdddd}. Instead of storing it directly as their ASCII values, we can store

{\centering\mono
    05 97 03 98 02 99 04 100
\medskip}

as there are 5 ASCII code 97s, 3 98s, 2 99s, and 4 100s, that correspond to 5×a, 3×b, 2×c, and 4×d. One can even compress images with this method. Consider figure \ref{fig:8x8image}. Its binary representation is shown here (1 is now black, and 0 is now white):

{\mono
    1 1 1 1 1 1 1 1\\
    1 0 0 0 0 0 0 1\\
    1 0 1 1 1 1 0 1\\
    1 0 1 0 0 1 0 1\\
    1 0 1 0 0 1 0 1\\
    1 0 1 1 1 1 0 1\\
    1 0 0 0 0 0 0 1\\
    1 1 1 1 1 1 1 1\\
\medskip}

We can represent this in RLE format as:

{\mono
    81\\
    11 60 11\\
    11 10 41 10 11\\
    11 10 11 20 11 10 11\\
    11 10 11 20 11 10 11\\
    11 10 41 10 11\\
    11 60 11\\
    81\\
\medskip}

which represents the same data, but now compressed (note that {\mono 81} means 8 1's). This also goes for colors, just substitute the different colors for numerical/binary values. Refer to page 37 in the textbook for more examples.

\end{document}
