\documentclass[../main.tex]{subfiles}
\graphicspath{{\subfix{../images/}}}

\begin{document}

There are several applications of the Hexadecimal number system in Computers, which include and are not limited to:

\begin{itemize}
    \item Making binary easier to write/represent
    \item OS error codes
    \item MAC addresses
    \item IPv6 addresses
    \item Color codes
\end{itemize}

\paragraph{Making binary easier to write/represent}

Given long binary sequences, such as {\mono 1011101010111110}, programmers may find it much easier to simply express the given example as {\mono BABE}, which is 4 characters and not 16 when typing/writing. Converting between the two formats was covered in \hyperref[sec:1-converting-between-bin-den-and-hexadecimal]{this section}.

\paragraph{OS error codes}

OS error codes are given when a program has an error and must exit. Sometimes, they are represented in hex.

\paragraph{MAC addresses}

These uniquely identifies a device on the network. They use 48 bits of data in total (6 segments of 2 nibbles), and look like {\mono AA:BB:CC:DD:EE:FF}. Refer to section \ref{2:sec:mac} for details.

\paragraph{IPv6 addresses}

Refer to \ref{2:sec:ipv6} for details.

\paragraph{Color codes}

Color codes are used to represent colors on a computer. They describe the amount that specific colors mix together. Red, Green and Blue are used as the base colors, as they are the additive color primary colors (not yellow!) like {\mono \#ffcc00}. 2 nibbles or one byte is used to represent the magnitude of the color itself, from 00-ff. The previous example consists of $FF$ red, $CC$ green and $00$ blue, which produces a yellow color.

\end{document}
