\documentclass[../main.tex]{subfiles}
\graphicspath{{\subfix{../images/}}}

\begin{document}

There are several applications of the Hexadecimal number system in Computers, which include and are not limited to:

\begin{itemize}

\item Making binary easier to write/represent

\item OS error codes

\item MAC addresses

\item IPv6 addresses

\item Color codes

\end{itemize}

\paragraph{Making binary easier to write/represent}

Given long binary sequences, such as 

\begin{minted}{c}
1011101010111110

#include <stdio.h>
int main(void) {
    printf("weewoo\n");
    return 0;
}

\end{minted}

Programmers may find it much easier to simply express the given example as {\mono hello world some text here}, which is 4 characters and not 16 when typing/writing.
Converting between the two formats was covered in \hyperref[sec:1-converting-between-bin-den-and-hexadecimal]{this section}.

\paragraph{OS error codes}
stub

\paragraph{MAC addresses}
stub

\paragraph{IPv6 addresses}
stub

\paragraph{Color codes}
stub

\end{document}
