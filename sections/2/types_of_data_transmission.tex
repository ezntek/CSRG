\documentclass[../main.tex]{subfiles}
\graphicspath{{\subfix{../../images/}}}

\begin{document}

There exists several modes of data transmission:

\begin{itemize}

\item \textbf{Serial Transmission:} This is the process of sending data through one single channel, one bit at a
    time. It is slow, but reliable for small amounts of data. This is used by USB, the kind of connector that is used for flash drives, most modern
    peripherals like mice and keyboards, and other devices.

    All generations of USB\footnote{There are multiple generations of USB, please check \hyperref[2:usb]{this section} for more information.} that
    make use of the USB-A conector use serial exclusively.

\item \textbf{Parallel Transmission:} This is the process of sending multiple bits at once, thorugh multiple channels concurrently\footnote{At the
    same time.}. This is faster, but it requires more computational resources, and can be prone to errors if one lane is congested\footnote{
    Thought problem: If parallel uses many data channels to send 4 bits, and the second channel is clogged somehow, will the data send, and will
    there be an error? The solution is trivial and explains this property.}. The data may also become skewed\footnote{Arrive out-of-order.}, and may
    cause unwanted issues.

\item \textbf{Simplex:} This is when data can only be sent one way.

\item \textbf{Half Duplex (HD):} This is when one device can send data at one time, like a walkie-talkie. The other device must wait until the data
    is fully sent; only then can they respond. This method is used in some radio communication protocols.

\item \textbf{Full Duplex (FD):} This is when data can be sent both ways at the same time. This is like a phone call or video chat, where both people can
    talk simultaneously. This is used in video conferences, phone chats and real-time web applications.

\end{itemize}

\end{document}
