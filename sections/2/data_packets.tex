\documentclass[../main.tex]{subfiles}
\graphicspath{{\subfix{../../images/}}}

\begin{document}

Sending data across large distances is used all the time, like in video chats, voice chats, uploading homework or simply loading HTML\footnote{This markup language, commonly mistaken for a programming language is what lays out websites; it dictates the structure of the site, like the bricks for a house.}. Typically, sending data like this is done through a stream of \textbf{data packets}, otherwise known as just packets.

A packet looks like the following:

\includegraphics[scale=0.5]{packet.png}

\newpage

\paragraph{Components of a packet}

\begin{itemize}
    \item The Header:
        \begin{itemize}
            \item The IP\footnote{IP Address; in this context, always IPv4 unless if specified} of the sender
            \item The IP of the receiver
            \item The sequence number; If a lot of data must be sent throughout multiple packets, the
                  \emph{sequence number} makes sure that the packets' payloads are reassembled in the correct order.
            \item The packet size, in order to make sure the packet received is of the correct size.
        \end{itemize}
    \item The payload, which consists of the binary data to be transmitted via the buffer.
    \item The trailer, which consists of:
        \begin{itemize}
            \item Some way of identifying the end of the packet. This is typically some special value like a null terminator\footnote{In programming,
                  this is referred to as a \emph{sentinel value}, which just means a value that carries some special meaning. An example is like in the
                  C programming language, to tell the end of a string, a sentinel value of 0 is used to denote the string finished.}. The algorithm can
                  then scan the data until it hit that character to extract the payload.
            \item An error checking method. CRCs are used to check this (see \hyperref[par:CRCs]{below}).
        \end{itemize}
\end{itemize}

\label{par:CRCs} % label used lol
\paragraph{Cyclic Redundancy Checks}





\end{document}
